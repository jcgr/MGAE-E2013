\section{Problems}
\label{04}

While the program can do simple rendering, there are a few problems. One of them arises when the camera is moved through the model and no longer looks at it. The results for the screenpoints will overflow and cause the program to crash. This could be avoided by updating only the triangles the camera is looking at.

Another problem happens when the camera is turned. At some point, the transforms will end up putting one of the closer to the oppesite side of the screen from where the rest of the model is, which means the triangle will stretch across the screen. A check to see where the rest of the triangle is would help in getting rid of this problem.

A smaller problem is performance-wise. Each triangle has three vertices that needs to have their screenpoints calculated. But some of these vertices are the same for the triangles (for example the top of the pyramid). This means that some vertices may be updated multiple times, resulting in a lot of extra calculations. This could be avoided by letting each triangle hold a reference to the vertices, instead of having them hold instances of the same vertex.